\documentclass[11pt]{article}   	% use "amsart" instead of "article" for AMSLaTeX format
\usepackage{geometry}                		% See geometry.pdf to learn the layout options. There are lots.
\geometry{letterpaper}                   		% ... or a4paper or a5paper or ... 
%\geometry{landscape}                		% Activate for rotated page geometry
%\usepackage[parfill]{parskip}    		% Activate to begin paragraphs with an empty line rather than an indent
\usepackage{graphicx}				% Use pdf, png, jpg, or eps§ with pdflatex; use eps in DVI mode
								% TeX will automatically convert eps --> pdf in pdflatex		
\usepackage{amssymb}
\usepackage{comment}
%SetFonts

%SetFonts




\begin{document}
\title{Characterization of Time Domain ToO Targets}
\author{Alex Kim}
\date{}							% Activate to display a given date or no date
\maketitle
\begin{abstract}
A DESI secondary observing plan may include Time Domain targets.  Tile Selection, Fiber Allocation, and bookkeeping must be
designed to accommodate these targets.  This note describes the Time Domain targets in order to inform the design
of these DESI processes.
\end{abstract}

\section{Time Domain ToO Targets}
There are several classes of Time Domain ToO Targets.  
\begin{itemize}
\item {\bf Galaxies in the BGS target list that do not yet have a redshift}

\begin{itemize}
\item BGS Tiles only.
\item Tile Selection: No influence.
\item Fiber Assignment: These galaxies given higher priority relative to other BGS targets.
\item Total $\sim 4.8$ targets per square degree.  85\% will have been identified by the start of the DESI survey, 100\% will be
identified by the end of the 3-year ZTF survey 6 months after the start of DESI.   Continued target production of this class at $\sim 1.6$ targets sq$^{-1}$yr$^{-1}$ will occur if there is an extension
of the ZTF survey.
\end{itemize}
\item {\bf Galaxies that are not in the BGS target list}

For most BGS-level depth is sufficient, some may require Survey depth.
\begin{itemize}
\item BGS, LRG, ELG Tiles.
\item Tile Selection: No influence. 
\item Fiber Assignment: Priority relative to BGS, LRG, ELG, Sky targets TBD.
\item Total $\sim 0.2$ targets per square degree.  85\% will have been identified by the start of the DESI survey, 100\% will be
identified by the end of the 3-year ZTF survey 6 months after the start of DESI.
Continued target production of this class  at $\sim 0.007$ targets sq$^{-1}$yr$^{-1}$ will occur if there is an extension
of the ZTF survey.
\end{itemize}
\item {\bf Active transients}

These supernovae are expected to be at $z \lesssim 0.15$.  Some require Survey depth.
\begin{itemize}
\item BGS, LRG, ELG Tiles.
\item Tile Selection: Tiles with active transient given priority.
\item Fiber Assignment: Priority relative to BGS, LRG, ELG, Sky targets TBD.
\item $\sim 0.03$ targets per square degree at any given moment.
\end{itemize}
The target lists, updated daily, contain SNe that are expected to be above nominal depth thresholds on that night.
 Graur and Moustakas have quantified extraction efficiency as a function
of magnitude for the BGS, with the cutoff at   $<20.5$~mag.
SNe will therefore enter and leave target lists as their light curves evolve. 
The duration a SN stays in the list depends on its distance
and absolute magnitude, generally on the time-scale of a week.

An object observed and classified by DESI does not have to be reobserved and is removed from the target list.
\end{itemize}

All target lists will be updated at $\sim 1$-day timescales.  For galaxy targets, one month latency between fiber assignment and latency is acceptable.
 For active-transient targets, $<$~few-day latency between fiber assignment and latency is required.

All target lists may have subpriorities determined by the Time Domain Working Group.  To first order, the subpriority will be defined by the
anticipated quality of (non-DESI)  photometry for the SN, and the expectation that a spectrum may be more efficiently obtained by another
spectroscopic resource.

The BGS magnitude limit is $r \sim 19.5$.  Most hosts will be brighter than this magnitude, some will be fainter.

\section{Tiles}
The default plan is to only consider targets within the DESI footprint with active tiles to be observed.  There may be Target
of Opportunities in non-DESI fields.  A custom tile and fiber-assignment file can be made.

\section{Considerations for Fiber Assignments}

There are two ways to implement a Fiber Assignment (FA) for a time-domain target.  The first is to modify the target list to include 
the desired object and priority/subpriority when FA is run.   If the FA is run $> 1$~week of the observation, this method
is satisfactory for galaxy targets but not active-transient targets.  If the FA is run $<1$~week of the observation, this method is 
satisfactory for all targets.  An implication is that the target list will evolve non-deterministically.

The second way is by reassigning fibers within an existing pre-generated FA file any time before the observation.
Quoting Stephen Bailey,
\begin{quote}
This would make bookkeeping easier because you know *exactly* what was changed, rather than relying upon an algorithm where the transient may bump target X that then gets picked up by a different fiber such that target Y got bumped, causing a change to target Z...  If implemented as a single fiber override, the bookkeeping for the normal problems becomes equivalent to a fiber that breaks for a single exposure and then returns.
\end{quote}

Things necessary for the expanded pipeline:
\begin{itemize}
\item New  Object Class or subclass ``bitfields'' for our targets.
\item Dynamic changes to the input catalog (this solution not preferred.
\item  Ability to update FA output (FITS) file and bookkeeping if our allocation occurs outside of the standard FA pipeline.
\end{itemize}


This document is available on github git@github.com:AlexGKim/desiztf.git

\begin{comment}
\section{Properties of the ZTF Survey}

\subsection{Survey area}
ZTF will survey the visible sky from Palomar with few day cadence, which 
sets a dec$>-30$ limit.
Most SN searches will skip the galactic plane, so galactic latidue $>~ 
13$. The galactic plane is (frequently) observed but is typically skipped 
for SN searches to limit stellar variability. As a nominal survey, every 
field visible at Palomar at dec$>-30$ and b$>13$ will be observed every 
~third day.
[We could make a figure of the sensitive area vs time during the year if 
needed.]

\subsection{ZTF SN search season}
Palomar winter weather is worse, which will both reduce search 
efficiency and lc quality.
We therefore assume 300 search days/year.
As with covering the galactic fields, the ZTF survey will be operating 
the full year and there will be SNe found also here.
Not assuming searching the full year also allows for some instrumental 
downtimes.

\subsection{What ZTF is  searching for}
Looking for efficient, complete SNIa program. Means looking for smoothly 
rising transients which are rising >7 days from the 20.5 ZTF detection 
limit and peaking at <19.5 mag and with multi-band lc compatible with 
SNIa. Translates into very efficient (pure) selection of transients for 
spectroscopic follow-up.
Loosing these requirements (detection, peak, lc-fit) would rapidly 
increase the number of candidates. This is not assumed here. For a large 
DESI program, especially looking for early transients or more complete 
samples this could make sense.


\subsection{Expected detection rates}
Based on a combination of the rate observed so far and simulations we 
expect ~10 transients that peak each night that fulfills the above 
requirements. 6-8 of these would be SNe Ia (the remaining mainly Ibc SNe 
and AGN). The SNIa sample would be expected to be normally distributed 
with a peak around ~0.08 and extending to z~0.12. Based on previous SN 
surveys we expect 90\% of the SNeIa to have host galaxies bright enough 
to be a potential DESI BGS target. (A fraction of the SNe Ia host 
galaxies will already have eg SDSS redshifts. It has been argued this 
should be as much as 30\% but in practice so far is has been much lower, 
more like 10\%).


\subsection{Full sample size and priority}
Scaling the above numbers yields 3000 potential transients / year of ZTF 
operation evenly distributed over the ZTF SN search area. These could be 
prioritized according to the following metrics:
- As mentioned, a fraction will already have some sort of redshift and 
would thus have lower priority. (remove completely?)
- Priority inversely weighted with LC quality fit. In the static regime 
of looking for host galaxy redshift the full transient lightcurve can be 
used for this, which would could give a very strong rejection of eg AGN 
variability.
- Live spectroscopy of hostless transients. Live spectroscopy required 
for their use, but would of course not have corresponding DESI target.
- SNe in particular regions of the sky, e.g. galaxies with previous 
transients or close to voids or high density regions.
- SNe in regions with additional follow-up, eg. I-band. (This could go 
both ways - we could tune the I-band coverage region to match the 
projected DESI fields).

If priorities were used, out of the 3000 candidates/year:
~100 have very high priority (particular hosts or hostless)
~1000 have high priority (spectroscopically typed SNe Ia where a host 
redshift is only missing component for cosmology)
~1000 "normal" priority (photometric SNeIa or typed but with spotty lc)
~1000 "low" priority (lc either not great or preexisting redshift)


Again, this would be the sample of interest for a SNIa peculiar velocity 
study. For eg. rates or completeness studies one would like to go 
deeper, which would easily increase sample size with an order of 
magnitude.
\end{comment}

\end{document}  